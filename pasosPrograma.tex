  ------------------------------------------------------------------------------------------------------------------------------------------------------------------------------------------------------------------------------------
                                        Pasos para el programa en Js
  ------------------------------------------------------------------------------------------------------------------------------------------------------------------------------------------------------------------------------------
  1-deberia iniciar el programa como esta el html5

  2- crea elementos de captura para asignarlos uno para fotos y uno para textos relacionados al secction2

  3- crear un array de validaciones de tipo caracter, esto con la idea de reconocer cuando una palabra ya esta encriptada o no.

  4- crear y llenar vector de 4 posiciones  este llena de imagenes referentes a un mismo tema, y de igual 
  forma un vector que contenga las mismas caracteristicas pero de tipo caracter, que contenga mensajes que 
  referencia a las imagenas del otro vector 
  
  5-hay que crear una funcion recursiva, que permita hacer una repeticion constante mientras no ocurra una accion del usuario
  
  6- crea una variable tipo caracter candado sin valor
 
  4- crear una condicion para verificar que esta contenido en el texarea principal y en dependencia de eso crear ciertas acciones:
    
      *acciones si no existe texto:
         - crear una funcion aleatorea que genere un numero el cual va a aestar dimencionado por cualquiera de los arrays ya creados, tome ese numero 
         y lo guarde en una variabble

         -llamar a la funcion aleatorea dos veces por cada condicion para que genero dos numero los cuales van a determinar la aleatoriedad de textos e imagenes

         - asignar el valor que contien especificamente los arrays con el numero aleatoreo y guardadrlos en dos variables, las cuales vamos a mostrar en nuetsra 
         web a traves del DOM en dos div que ya estan creados y estilizados,

         crear una funcion de temporizador que permitira dejar las imagenes y el texto por un tiempo determinado y solicitar llamar la funcion principal para hacer la recursividad requerida

      *acciones si existe texto entonces :

        -llamar a la funcion aleatorea dos veces por cada condicion para que genero dos numero los cuales van a determinar la aleatoriedad de textos e imagenes referentes a si hay un movimiento usuario 

        - asignar el valor que contien especificamente los arrays con el numero aleatoreo y guardadrlos en dos variables, las cuales vamos a mostrar en nuetsra 
        web a traves del DOM en dos div que ya estan creados y estilizados

        - crear un condicional, para realizar las acciones de eincriptado o desencriptados segun parametro expresiones de validadcion

        - la expresion a evaluar es la siguiente:
            + creamos un for para recorrer el array de validaciones ay comparar si la variable vandado contiene 
           for(let i=0; i < clave.length; i++){
              if(ingreso.includes(clave[i])){ 
              }        
          }

          *acciones si ingreso incluye validaciones:
             
              a- desabilitar el boton de codifique
              b- hacer un condicional para determinar si el boton decodifica fue presionado o si el usuario realizo la accion de decodificar 

              *acciones si presiono boton:
                 -llamar a la funcion desencritado();
                 -dentro de esta sintaxis tambien se debe mostrar el resultado en el texarea2
                 -limpiar el texarea1
                 -dentro de la funcion de desencriptado debe estar las sintaxis similiar a la de la funcion repetir para ejecutaracciones similares, como las llamada a la funcion generear numero aleatoro, esto para colocar la imagen y el texto relacionado al desencriptado, en este pedazo de codigo.
                 -se llama a la  funcion de temporizacion(); pero en este caso para asigan caracteres vacios ("") al texarea2 para hacer una limpieza despues de cierto tiempo
                 -llamar funcion repetir();
              
              *acciones si no presiono boton:
                -llama a la funcuion temporizador() repetir y se ajusta el tiempo a 3000;
          
         *acciones si ingreso no incluye validaciones:

              a- desabilitar el boton de decodifique
              b- b- hacer un condicional para determinar si el boton codifica fue presionado o si el usuario realizo la accion de codificar 

              *acciones si presiono boton:
                  - llamar a la funcion encritado();
                  - limpiar el texarea1
                  - dentro de la sintaxis de la funcion encriptado() debemo agregar el siguiente codigo que va a permitir que la frase ecriptada contenga una validacion:

                      a. llamar funcio aleatorea();y el resultado guardarlo en una varable
                      b. crea una variable de nombre candado a la cual se la va a asignar el array de validaciones conteniendo la posicion con el numero aleatorio
                      c.crea otra varable de nombre endriptado a la cual se le va a asignar los valores concatenados de candado + ingreso, de esta forma la variable encriptado generara un valor de ese contexto

                  - dentro de esta sintaxis tambien se debe mostrar el resultado en el texarea2
                  - dentro de la funcion de encriptado debe estar las sintaxis similiar a la de la funcion repetir para ejecutaracciones similares, como las llamada a la funcon generear numero aleatori, esto para colocar la imagen y el texto relacionado al encriptado, en este pedazo de codigo.
                  - se llama a la  funcion de temporizacion(); pero en este caso para asiganr caracteres vacios ("")  al texarea2 para hacer una limpieza despues de cierto tiempo
                  - llamar funcion repetir();

              *acciones si no presiono boton:
                    - llama a la funcuion temporizador() repetir y se ajusta el tiempo a 3000;
------------------------------------------------------------------------------------------------------------------------------------------------------------------------------------------------------------------------------------
                                                         ULTIMOS CAMBIOS VIERNES 8
------------------------------------------------------------------------------------------------------------------------------------------------------------------------------------------------------------------------------------
            - en la LINEA 286 se solicita habilitar el boton de encriptado nuevamente para corregir un posible error de bloqueo de ambos botones a la vez 
            ya que este boton es bloqueado despues que la validacion ejecuta una accion, y no vuelve a desabilitar a menos que se llame a la funcion repetir();
            por ende en este punto se va a hacer un llamdo a la funcion repetirencriptado(); y en este punto para encriptae el boton deberia estar ahabilitado 
            
            
                    "  //habilitacion del boton codificar para codificaion de una frase 
                        document.getElementById("codifique").disabled = false;
                        console.log("deberia haber habilitado el boton de codificacion ");

                        //en este punto se debe desabilitar el boton de desencriptar ya que esta habilitado para evitar su accionar se bloquea nuevamente
                        
                        //desabilitacion del boton decodificar para evitar recodificaion de una frase 
                        document.getElementById("decodifique").disabled = true;
                        console.log("deberia haber desabilitado el boton de decodificacion ");  

                        repetirencriptado()  "

        
            - En la LINEA 335 de prueva.js, colocamos este codigo que ejecuta la accion de validar si en este punto el usuario borro o pego una frase, como dicha frase 
             no se sabe que va a ser, se realizan comparaciones para este punto y se envian acciones de respuesta segun la peticion, este codigo es un repetido en varios 
             lugares del programa, de hecho aqui tambien se hace la solictud de bloque y desbloqueo de botones 


                   
                        " //se hace otra comparativa para saber si el texarea1 esta limpio y volver al principio del codigo
                        if (repitedesencripta1 === "") {
                        console.log("deberia haber llamado a repetir() con retardeo de 1seg ");
                        //funcion de temporizador repetir para volver al inicio despues de 2seg
                        setTimeout(() => repetir(), 1000);

                        //este sino es el de la comparativa de repitedesencripta1 texarea vacia;
                        }else{ 

                            //se llama a la funcion de validaciones para saver si el texto sigue siendo un texto a desencritar, y por parametros le vamos a pasar el valor del texarea1 que esta en este momento
                            if(validacion(repitedesencripta1) === "desencriptar"){
                                console.log("estoy en condicional de validacion esta validacion es verdadera")

                                //imagenes y textos aleatoreos indicanto que si existe actividad usuaria por el lado de encriptaciones 
                                const eleimgvida = generarAleatorio(HayVidaimg.length)
                                const eletexvida = generarAleatorio(HayVidaTex.length)
                                let imgaleatory1= HayVidaimg[eleimgvida];
                                console.log("mostrando el valor de imagen aleatorea ", imgaleatory1);
                                let texaleatory1 = HayVidaTex[eletexvida];
                                console.log("mostrando el valor de los textos ", texaleatory1)

                                //asignando los valores a los divs aleatoreos
                                divImag.src = imgaleatory1;
                                divText.innerHTML = texaleatory1;
                        
                                //funcion de temporizador recursivo
                                setTimeout(() => repetirdesencripta(), 3000);

                                //este sino es la parte falsa de validaciones de repetirdesencripta()
                            } else{
                                console.log("estamos en comparativa falsa de la funcion  repetirdesencripta(), la variable repetirdescritiva es;",repitedesencripta);
                                
                                //habilitacion del boton codificar para codificaion de una frase 
                                document.getElementById("codifique").disabled = false;
                                console.log("deberia haber habilitado el boton de codificacion ");

                                //en este punto se debe desabilitar el boton de desencriptar ya que esta habilitado para evitar su accionar se bloquea nuevamente
                                
                                //desabilitacion del boton decodificar para evitar recodificaion de una frase 
                                document.getElementById("decodifique").disabled = true;
                                console.log("deberia haber desabilitado el boton de decodificacion ");  
                            
                            
                                //se supone que el texto ingresado en este punto es una frase para encriptar por ende se llamara a la funcion repetirencripta();
                                repetirencripta();
                            }
                        } "

            - En la LINEA 443 de prueba.js se pega este pedaxo de codigo que va a ejecutar la accuon contraria al codigo que esta en la LINEA 286 de prueva.js lo cual es habilitar y desabilitar botones 
                            
                                "//desabilitacion del boton codificar para que no hacer recodificaion de una frase 
                                document.getElementById("codifique").disabled = true;
                                console.log("deberia haber desabilitado el boton de codificacion ");

                                //en este punto se debe habilitar el boton de desencriptar ya que esta frase es un encriptado
                                //habilitacion del boton decodificar 
                                document.getElementById("decodifique").disabled = false;
                                console.log("deberia haber habilitado el boton de decodificacion ");  
                                    
                                //se supone que el texto ingresado en este punto es una frase para desencriptar por ende se llamara a la funcion repetirdesencripta();
                                repetirdesencripta();"

            - En la LINEA 

--------------------------------------------------------------------------------------------------------------------------------------------------------------------------------------------------------------------------------------------
                                                                     LUNES 11/3/2024
--------------------------------------------------------------------------------------------------------------------------------------------------------------------------------------------------------------------------------------------
    se realizaron una serie de prubeas y arreglos, entre ellos;
    - se creo un repositorio local para compartir los archivos

    * En el archivo html
      - en las lineas 19-20 se aqgrego una etiqueta <a> para linquear la cuenta de quien realizo el proyecto y que puedan observar el codigo fuente
      - en la line 52 se agrego una etiqueta de <label> para asignarle al usuario que es un boton de copiado porque el boton tiene un nombre tematico 
      - tambien se asigo un emoji al boton de copiado que indica que ese es su usuario
      -en la linea 96 existia un / que estaba interfiriendo en la ruta de apertura del proyecto error 404

    * En el Archivo CSS
      - en la linea 63 se le da estilo a la etiqueta <a> color especificamente
      - en la linea 110 se le realizo un cambio al hoover del texare1 se cabio el cursor: de pointer a text, generaba confucion 
      - en las lineas 218,235 y 293 se eliminaron / y se asignaron "", para que el servidor encontrara las imagenes relacionadas, generaba un error 404
      - en la linea 302 se realiza el mismo trabajo que en la linea 110, pero en texarea2
      - en la linea 337 se aplican estilos a la etiqueta label 
    
    * En el Archivo Js
      - desde l alinea 32-37 se corrige un array de imagenes con ruta mala por un /, generava 404
      - linea 285 se agregan los numeros a la funcio de reconocimiento de caracteres, no estaba reconociendo los numeros 
      - linea 389 se agrega codigo para desabilitacion del boton desencriptado ya que estaba ejecutando la accion de una encriptaciones
      - linea 441 se agrega codigo para la habilitacion del boton desencriptado porque este se desabilita en un sector arriba n el codigo 
      - linea 587 se agrega codigo para desabilitacion del boton encriptado ya que estaba ejecutando la accion de una encriptaciones
      - linea 654 se agrega codigo para habilitacion del boton encriptado ya que estaba ejecutando la accion de una encriptaciones
      - se realiza la eliminacion de console.log innecesarios en el codigo al rededor de 75 que se encontraban entre la linea 121 y la 717
    
------------------------------------------------------------------------------------------------------------------------------------------------------------------------------------------------------------------------------------
                                                         MIERCOLES 13/3/2024
------------------------------------------------------------------------------------------------------------------------------------------------------------------------------------------------------------------------------------
        -Se crea el archivo README
        - Se realizan pruebas y commits de varios cambios 
        - Se crea un repositorio remoto y local.
        - Se crea la direccion web del proyecto https://miguel7g.github.io/challenge-encriptador7g/
        - Se crea dos topics
        - Se publica en el ¡formulário del Alura Challenge!
       
          